\documentclass[12pt]{article}
\usepackage{cclicenses}
\title{Tutorial on Tables and Figures}
\author{Kannan Moudgalya \\ kannan@iitb.ac.in \\ \byncsa}
\date{\today}

\begin{document}
\maketitle
\newpage

\begin{tabular}{||l|c|c|c|r|}\hline
\multicolumn 2 {||c|}{Fruit details} & 
\multicolumn 3 {c|}{Cost calculations} \\ \hline
Fruit & Type & No. of units & cost/unit & cost (Rs.) \\ \hline
Mango & Malgoa & 18 & 50 &  \\ \cline{2-4}
      & Alfonso & 2 & 300 & 1,500 \\ \hline
Jackfruit & Kolli Hills & 10 & 50 & 500 \\ \hline
Banana & Green & 10 & 20 & 200 \\ \hline
\multicolumn 4{||r|}{Total cost (Rs.)} & 2,200 \\ \hline
\end{tabular}
\end{document}


















Mango, jackfruit and banana are native fruits to India.  As a matter
of fact, the word Mango comes from similar words in Tamil and
Malayalam.  There is a mention of these three fruits in the ancient
Tamil literature: these are known as THE three fruits in Tamil. 

According to archeological findings, jackfruit has been cultivated 
in India for over three thousand years.
Jackfruit is grown in South and Southeast Asia too.  It is the
national fruit of Bangladesh and Indonesia.  It is also grown in
several other tropical areas, such as, central and eastern Africa,
islands of the West Indies, Brazil and Surinam.

